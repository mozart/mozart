\section{Non-Structural Subtyping Constraints}

A \emph{signature} $\Sigma$ is a set of symbols where every symbol
$f\in \Sigma$ has an \emph{arity} $\alpha_\Sigma(f) \in \N$. A
\emph{partially ordered signature} is a signature $\Sigma$ equiped
with a partial order $\lleq_\Sigma$. An \emph{partially ordered
  signature} is \emph{convex} if $f \lleq_\Sigma g \lleq_\Sigma h$
implies that $\alpha_\Sigma(g) \geq
\min(\alpha_\Sigma(f),\alpha_\Sigma(h))$. Otherwise said, in a convex
signature no symbol is situated between to symbols with strictly
larger arities. A \emph{(C)POS} is a (convex) partially ordered signature.
We usually ommit the index $\Sigma$ to $\alpha$ and $\leq$ when the
context is unabigeous.

A \emph{tree domain} is a (possibly infinite) non-empty subset $D$ of
$\N^*$ which is
\begin{itemize}
\item \emph{prefix-closed}, that is if $\pi\pi'\in D$ then $\pi \in D$
\item \emph{left-closed}, that is if $\pi (i+1)\in D$ then $\pi i\in D$.
\end{itemize}

Let $\Sigma$ be a POS. A \emph{$\Sigma$-tree} is a pair of a tree
domain $D$ and a labelling function $L\colon D \rightarrow \Sigma$
such that for all $\pi in D$ we have that $\alpha(\L(\pi)) = \max\{i
\mid \pi i\in D\}$. For a given tree $\tau$ we write $D_\tau$ for its
domain, and $L_\tau$ for its labelling function.

For a given POS $\Sigma$, the \emph{non-structural subtyping relation}
$leq$ on the set of $\Sigma$-trees is defined by:
\[
s \leq t \qquad \Leftrightarrow \qquad
\forall \pi \in D_s \cap D_t :  L_s(\pi) \lleq L_t(\pi)
\] 
Obviously, the relation $\leq$ is reflexive and anti-symmetric.

\begin{lemma}
  Given a POS $\Sigma$, the relation $\leq$ is transitive (hence, a
  partial order) iff $\Sigma$ is convex.
\end{lemma}
\begin{proof}
  Let $s \leq t \leq u$. We first show that $D_s \cap D_u \subseteq
  D_t$. From this the transitivity of $\leq$ follows immediately,
  since for any $\pi\in D_s \cap D_u$ we have that $L_s(\pi) \lleq
  L_t(\pi) \lleq L_u(\pi)$, hence $L_s(\pi) \lleq L_u(\pi)$ by
  transitivity of $\lleq_\Sigma$.

  Assume that $\pi \in D_s \cap D_u$ and $\pi \not\in D_t$. Let $\pi'$
  be the longest prefix of $\pi$ with $\pi'\in D_s \cap D_u \cap D_t$
  (this is well-defined since every tree domain contains
  $\epsilon$). Let $i$ be the natural number such that $p'i$ is a
  prefix of $\pi$. It follows that $\alpha(L_s(\pi')) \geq i$ and
  $\alpha(L_u(\pi')) \geq i$, but $\alpha(L_t(\pi')) < i$. Since
  $L_s(\pi') \lleq L_t(\pi') \lleq L_u(\pi')$ this contradicts the
  convexity of $\Sigma$.

  Let $\Sigma$ be non-convex, that is $f \lleq g \lleq h$ with
  $\alpha(g) < \min(\alpha(g),\alpha(h))$. Let $r$ be some arbitrary
  $\Sigma$-tree. Define
  \begin{eqnarray*}
    s & := & f(\underbrace{r,\ldots,r}_{i \mbox
      { times}},h(r,\ldots,r),\ldots h(r,\ldots,r))\\
    t & := & g(\underbrace{r,\ldots,r}_{i \mbox
      { times}})\\
    u & := & h(\underbrace{r,\ldots,r}_{i \mbox
      { times}},f(r,\ldots,r),\ldots f(r,\ldots,r))\\
  \end{eqnarray*}
  It is easy to see that $s \leq t$ and $t \leq u$, but $s \not\leq u$.
\end{proof}

\emph{Non-structural subtyping constraints} over a signature $\Sigma$
are defined by the following grammar
\[
\phi ::= x \leq f(x_1,\ldots,x_n) \mid x \geq f(x_1,\ldots,x_n) \mid
\phi \und \phi
\]
where $x$ ranges over an infinite set of variables and $f$ over
$\Sigma$, and where it is understood that $n \geq 0$ is the arity of
$f$.  We write $f$ instead of $f()$, and use $s \geq t$ as synonyme for
$t \leq s$.

